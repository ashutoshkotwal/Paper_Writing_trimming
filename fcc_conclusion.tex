\section{Conclusions}
The studies presented in this paper show that the reconstruction of jet substructure 
variables for future particle colliders will benefit from small cell sizes of the hadronic calorimeters. 
This conclusion was obtained using the realistic \GEANTfour simulation of calorimeter responses combined with reconstruction of 
calorimeter clusters used as inputs for jet reconstruction. 
Hadronic calorimeters that use the cell sizes of $20\times 20$~cm$^2$ ($\Delta \eta \times \Delta \phi = 0.087\times 0.087$) 
are least performat almost for every 
substructure variables considered in this analysis for jet transverse momenta between 2.5 to 10~TeV. 
Such cell sizes are close to 
those used for the ATLAS and CMS detectors at the LHC. 
In terms of the reconstruction of the physics-motivated quantities  
used for jet substructure studies, the  performance 
of a  hadronic callorimeter  with 
$\Delta \eta \times \Delta \phi = 0.022\times0.022$ is, in most cases,
better than for a detector with  $0.087\times 0.087$ cells.
Thus this study confirms the  baseline SiFCC detector geometry \cite{Chekanov:2016ppq}
that uses $\Delta \eta \times \Delta \phi = 0.022\times0.022$ HCAL cells.
The performance of the HCAL with cells $\Delta \eta \times \Delta \phi = 0.0087\times 0.0087$ and 
$\Delta \eta \times \Delta \phi = 0.0043\times 0.0043$ were found to be similar.

It interesting to note that,  for very boosted jets with transverse momenta close to 20~TeV, no significant improvement with the 
decrease of cell sizes was observed. This result needs to be understood in terms of various type of simulations and 
different options for construction of the calorimeter clusters.


