\section{Conclusions}
The studies presented in this paper show that the reconstruction of jet substructure
variables for future particle colliders will benefit from small cell sizes of the hadronic
calorimeters. This conclusion was obtained using the realistic Geant4 simulation of
calorimeter responses combined with reconstruction of calorimeter clusters used as in-
puts for jet reconstruction. Hadronic calorimeters that use the cell sizes of 20�20 cm2
are least performat almost for every substructure variables considered in this analysis
for jet transverse momenta between 2.5 to 10 TeV. Such cell sizes are close to the nu-
clear interaction length of the considered calorimeter, and are similar to those used for
the ATLAS and CMS detectors.\\

It is however interesting to note that for very boosted jets with transverse moment
close to 20 TeV, no significant improvement for with decrease of cell sizes was observed.
This result still needs to be understood in terms of verious type of simulations and
different options for construction of calorimeter clusters.

%%%%%%%%%%%%%%% commented out 
\end{comment}

